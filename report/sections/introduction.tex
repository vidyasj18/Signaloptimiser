\chapter{Introduction}
\label{ch:introduction}

\section{Background and Significance}
\label{sec:background}

Traffic in growing Indian cities like Mangalore is complicated—mixed vehicle types, rising ownership, and roads that weren’t built for today’s demand. Intersections are where this pressure really shows. When signals aren’t tuned well, you get long queues, wasted fuel, higher emissions, and a dip in safety. Too often, plans are either outdated or set by intuition instead of data, which makes flows unpredictable and commutes frustrating.

This project tackles that head-on with a practical, data-driven approach to optimizing signal timings at key intersections. We combined automated video-based counts (YOLOv8), standard engineering practice (Webster’s method), and SUMO microsimulation to design timings that fit on-the-ground realities. The payoff is twofold: immediate gains (lower delay, smoother flow) and a repeatable way for cities to move from reactive fixes to proactive, evidence-based signal management. A simple Streamlit app ties it all together so the workflow is usable, not just theoretical.

\section{Problem Statement}
\label{sec:problem}

Intersections in Mangalore regularly face heavy congestion, long delays, and uneven traffic movement. A big part of the problem is suboptimal timings and manual control that can’t easily adapt to changing demand. Where signal plans do exist, they’re often out of date and don’t reflect current patterns, leading to inconsistent and unpredictable flows.

The result is familiar to anyone who uses these roads: longer queues, more stop-and-go, more emissions, and avoidable stress. Without a sound, data-backed basis for managing intersections, it’s hard for authorities to deliver efficient and fair mobility across the network.

\section{Main Objectives of the Project}
\label{sec:main_objectives}

\begin{enumerate}
    \item Select and classify two critical intersections in Mangalore (Jyoti Circle and Hampankatta Circle) for detailed study
    \item Collect field data via video and automated YOLOv8 counts, converting vehicle detections to Passenger Car Units (PCUs)
    \item Document site conditions, demand distribution, and congestion patterns across peak and off-peak hours
    \item Apply Webster’s method to compute cycle lengths and allocate green time
    \item Build a Streamlit app that integrates YOLO counting, Webster timing, and SUMO validation in one place
    \item Quantitatively evaluate performance (e.g., average delay, queueing)
    \item Validate optimized timings in SUMO under realistic conditions
    \item Produce clear phasing diagrams and visuals for the recommended plans
    \item Record workflow, field challenges, and adaptations to protect data quality
    \item Provide evidence-based recommendations for upgrades and operations
\end{enumerate}

\section{Scope and Limitations of the Project}
\label{sec:scope_limitations}

\subsection{Scope of the Project}

This work covers the end-to-end optimization of signal timings at two key intersections in Mangalore. We combine manual checks with video analytics to quantify volumes and movements, apply standard engineering (Webster’s formula) and supporting models to design cycles and green splits, and create clear visuals (phasing diagrams and charts) to communicate the plans. The outcome is meant to be reusable at similar sites and helpful to agencies aiming for practical, data-led improvements.

\subsection{Limitations of the Project}

A few constraints are worth noting. We studied only two intersections, so results may not generalize to all contexts. Data came from a limited observation window and may miss seasonal or event-driven spikes. Weather, equipment limits, and occasional manual error can affect counts. Real-time adaptive control was outside scope due to time and resources, so recommendations come from periodic data rather than continuous monitoring. Finally, factors like pedestrian behavior, informal street use, and enforcement were considered as observed, but they can influence real-world outcomes.

\section{Research Methodology and Approach}
\label{sec:methodology_approach}

\begin{enumerate}
    \item Select two representative intersections in Mangalore: one signalized (Jyoti Circle), one manually controlled (Hampankatta)
    \item Record peak and off-peak traffic using an Insta360 camera
    \item Run YOLOv8 to detect, classify, and count vehicles; convert to PCUs per approach
    \item Tabulate volumes and movement patterns for each approach
    \item Apply Webster’s method (per IRC:106-1990) to compute cycle length and green splits
    \item Use a Streamlit app to integrate detection, optimization, and simulation
    \item Produce optimized phasing diagrams and visual summaries
    \item Note field challenges and describe adaptations used to maintain data quality
    \item Follow IRC specifications for design and calculations throughout
    \item Provide clear recommendations and documentation for implementation
\end{enumerate}

\newpage

