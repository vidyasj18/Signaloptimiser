\chapter{Introduction}
\label{ch:introduction}

\section{Background and Significance}
\label{sec:background}

Traffic in fast-expanding Indian cities such as Mangalore is messy - different kinds of vehicles share roads, numbers keep going up owning things, yet paths never meant for how busy we've gotten now - spots where streets cross show it best stress becomes obvious. If signals don't sync right, lines stretch out, gas gets burned, more pollution, yet less protection. Plans frequently feel old-fashioned or get decided without care gut feeling over numbers, so things run shaky yet rides feel annoying.

This project takes aim right away - using real-world info to handle it, while skipping guesswork altogether tuning traffic light patterns at busy crossings - using video data that tracks vehicles automatically (YOLOv8), regular engineering approach - Webster's technique - or SUMO microsimulations help shape schedules that match real-world conditions. This brings two clear benefits: right away benefits like less wait time, easier movement - alongside a method places can use over again to shift from just responding adjustments to active, data-driven signal handling - a basic Streamlit interface connects everything side by side, making sure the process actually works instead of staying on paper.

\section{Problem Statement}
\label{sec:problem}

Choked roads in Mangalore often sit still for hours, held up by constant vehicle buildup plus unpredictable flow patterns movement. One key issue? Poor timing plus human-led adjustments that just don't quickly adjust when needs shift. In places with signal setups, those are usually outdated but they miss today's trends, causing uneven or erratic movement.

The outcome feels all too real for folks driving here - endless lines, frequent halts, or skip, extra pollution, plus preventable strain. With no noise, evidence-supported foundation for handling crossroads, officials struggle to provide smooth + balanced transport through the network.

\section{Main Objectives of the Project}
\label{sec:main_objectives}

\begin{enumerate}
    \item Select and classify two critical intersections in Mangalore (Jyoti Circle and Hampankatta Circle) to explore it closely
    \item Grab footage from the site, use YOLOv8 to tally vehicles automatically - then turn that into usable data detections to Passenger Car Units (PCUs)
    \item Record the layout of the place, how needs spread out, also where bottlenecks happen throughout high-demand times along with quieter periods
    \item Use Webster's approach to figure out how long cycles should be - then split up the green light periods accordingly
    \item Build a Streamlit app that integrates YOLO counting, Webster timing, and SUMO checking done just somewhere
    \item Check how well it works by looking at things like usual wait times or line lengths
    \item Check tuned schedules in SUMO using real-world scenarios
    \item Draw easy-to-understand step-by-step sketches along with visuals showing the suggested approach
    \item Log how tasks unfold, issues on site, also changes made - keep info reliable
    \item Show proof-backed tips for updates or running things
\end{enumerate}

\section{Scope and Limitations of the Project}
\label{sec:scope_limitations}

\subsection{Scope of the Project}

This project looks into improving signal times from start to finish at a pair of major crossroads - tweaking one affects the other, so both were adjusted together using real-world traffic patterns over in Mangalore. We use hands-on inspections along with video tools to measure how much stuff is there motions, use regular engineering methods - like Webster's equation - or related frameworks to design stages plus eco-friendly divisions - then build straightforward images like timing maps or graphs to talk about the plans - what's created should work again elsewhere, also useful for teams focused on real-world, evidence-based upgrades.

\subsection{Limitations of the Project}

A couple limitations should be mentioned. The research looked at just two crossing points, meaning findings might not apply across every situation. The info was gathered during a short period, so it might overlook weather shifts or gear issues sometimes cause surges - plus human mistakes pop up now and then might change the numbers. Handling live adjustments wasn't possible - ran out of time plus lacked tools, so suggestions pop up now and then since they're based on spaced-out info instead of constant tracking. In the end, factors such as how people walk around, casual use of sidewalks, also local rules being applied got taken into account seen, yet might still shape what happens in life.

\section{Research Methodology and Approach}
\label{sec:methodology_approach}

\begin{enumerate}
    \item Select a couple of typical crossings in Mangalore - pick Jyoti Circle, which has traffic lights, one run by hand (Hampankatta)
    \item Track high and low times with an Insta360 cam
    \item Launch YOLOv8 so it spots, sorts, then tallies cars - switch results into PCUs by direction
    \item List amounts along with flow trends per method
    \item Use Webster's approach from IRC:106-1990 to figure out how long the cycle should be - also work out the green time splits
    \item Run a Streamlit app that links detection with optimization while adding simulation
    \item Create tuned step-by-step layouts along with clear visual overviews
    \item Note the hurdles in the field - explain adjustments made to keep data reliable despite them
    \item Stick to IRC rules when designing or figuring stuff out every step of the way
    \item Pick straightforward tips along with easy-to-follow guides for putting things into action
\end{enumerate}

\newpage
