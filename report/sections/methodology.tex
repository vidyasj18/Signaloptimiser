\chapter{Methodology}
\label{ch:methodology}

\section{Objective Definition}
\label{sec:objective_definition}

The aim here is straightforward: build a data-driven, largely automated framework to optimize signal timings at urban intersections. Specifically, we set out to:

\begin{itemize}
    \item \textbf{Data Collection \& Intersection Selection:} Identify representative signalized and non-signalized intersections in Mangalore for a comparative case study, with practical field-feasibility in mind.
    
    \item \textbf{Automated Traffic Analysis:} Build a computer vision pipeline using YOLO to automatically count vehicles from recorded video and convert them to Passenger Car Units (PCU).
    
    \item \textbf{Signal Timing Optimization:} Use the extracted PCUs to compute cycle length and green splits for each approach, primarily via Webster’s method.
    
    \item \textbf{Validation \& Visualization:} Validate results with clear metrics (e.g., delay) and produce intuitive phase diagrams to explain the recommended cycle.
\end{itemize}

\section{Selection of Study Intersections}
\label{sec:intersection_selection}

We started by shortlisting intersections in Mangalore using Google Maps/Earth and field visits. We looked for a mix of traffic density, presence/absence of signals, and places where cameras could be set up safely and effectively.

Two intersections were finalized:
\begin{itemize}
    \item \textbf{Jyoti Circle} -- A busy three-arm junction, manually controlled by traffic police.
    \item \textbf{Hampankatta Circle} -- A four-arm intersection with non-functional signals and some movements restricted by barricades.
\end{itemize}

These locations are both practical to study and representative of common urban conditions.

\section{Field Data Collection}
\label{sec:field_data}

\subsection{Initial Plan}
We first planned to deploy multiple tripod-mounted cameras to cover each leg of both intersections.

\subsection{Challenges Faced}
Fieldwork rarely goes exactly to plan. In our case:
\begin{itemize}
    \item Continuous heavy rainfall
    \item Limited availability of tripods
\end{itemize}

\subsection{Revised Data Collection Setup}
To work around this, we used a single 360° Insta360 camera per site. That gave us:
\begin{itemize}
    \item Complete intersection coverage
    \item Reduced manpower
    \item Lower risk of missing turning movements
\end{itemize}

Each intersection was recorded for 30 minutes and the footage processed for analysis.

\section{Development of Analytical Framework in Python}
\label{sec:python_framework}

\subsection{Streamlit Web Application Development}

We built a Streamlit web app to tie the full workflow together. It:
\begin{itemize}
    \item Vehicle detection and PCU calculation using YOLOv8
    \item Optimum cycle time calculation using Webster's Method
    \item Green split allocation for each approach
    \item Phase diagram visualization
    \item SUMO simulation execution and metrics extraction
\end{itemize}

The interface avoids command-line steps and makes the process approachable for engineers and planners.

\section{YOLO-Based Vehicle Detection and PCU Estimation}
\label{sec:yolo_detection}

A two-stage computational system was built for vehicle detection and signal optimization.

\subsection{Stage 1: Streamlit-YOLO Application}

Users upload traffic videos from each approach via the Streamlit app. The app:
\begin{itemize}
    \item Vehicle detection using YOLOv8
    \item Tracking of inbound vehicles using ROI masks
    \item Conversion of detected counts into Passenger Car Units (PCUs) using IRC factors
    \item Export of PCU values into a unified JSON file
\end{itemize}

This JSON output feeds the signal timing calculations.

\subsection{Stage 2: Signal Timing and Phase Diagram Generator}

Signal timing is computed directly in the app:
\begin{itemize}
    \item Reads PCU data from the JSON output
    \item Computes optimum cycle time using Webster's method
    \item Calculates effective green, amber, and red times for NS and EW phases
    \item Generates a complete phase diagram using Plotly to visualize the signal sequence
    \item Validates proportional green allocation based on actual demand
\end{itemize}

The phase diagram shows the sequence and duration of green, amber, and red, ensuring NS and EW groups never overlap.

\section{Case Study Analysis for Selected Intersections}
\label{sec:case_study}

The system adapts to different intersection types by detecting whether a site is 3-way (T-junction) or 4-way (crossroads). It then generates the right network, routing logic, and timing approach for that geometry.

\subsection{Jyoti Circle (3-Arm T-Junction)}

Total PCU was about 6,802 with a strongly dominant NS flow. The system detected a 3-way layout (no East approach) and:
\begin{itemize}
    \item Generated appropriate T-junction network with only NB, SB, and WB approaches
    \item Applied T-junction routing logic (no through movements where not applicable)
    \item Predicted minimal cycle length ($\sim$60 s) appropriate for the traffic volume
    \item Allocated green split skewed toward the dominant direction (NS $\approx$ 36.5 s vs W $\approx$ 9.8 s)
\end{itemize}

Dynamic network generation resulted in 4 nodes, 6 edges, and 7 connections for the T-junction geometry.

\subsection{Hampankatta Circle (4-Arm Intersection)}

Total PCU reached 10,680 with heavy bus traffic. The system detected a 4-way intersection and:
\begin{itemize}
    \item Generated complete 4-way network with all approaches (NB, SB, EB, WB)
    \item Applied standard routing logic with through movements available
    \item Predicted longer cycle length ($\sim$92 s) to handle higher saturation
    \item Allocated balanced green splits according to proportional PCU distribution
\end{itemize}

The generated network reflected a full crossroads geometry.

Both cases show the model adapting to demand and geometry without manual configuration.

\section{Validation and Interpretation}
\label{sec:validation}

We cross-verified signal timing calculations against:
\begin{itemize}
    \item Webster's analytical cycle time formula
    \item Phase-wise saturation flow considerations
    \item PCU-based proportional green time distribution
    \item IRC:106-1990 standards for signalized intersections
\end{itemize}

We then validated plans in SUMO microsimulation:
\begin{itemize}
    \item Dynamic network generation for 3-way and 4-way intersections based on detected approaches
    \item Traffic light program creation from Webster signal timing plans
    \item Route generation based on PCU values from field data
    \item Running simulations under realistic traffic conditions
    \item Extracting performance metrics including average delay, waiting time, travel time, throughput, and time loss
\end{itemize}

The simulation provided detailed metrics supporting the Webster-based plans:
\begin{itemize}
    \item Average vehicle delay per vehicle
    \item Average waiting time at intersections
    \item Average travel time through the intersection
    \item Vehicle throughput (vehicles per hour)
    \item Total time loss compared to free-flow conditions
\end{itemize}

In short, the microsimulation backs up the effectiveness of the optimized timings under realistic conditions.

\newpage

