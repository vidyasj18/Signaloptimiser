\chapter{Methodology}
\label{ch:methodology}

\section{Objective Definition}
\label{sec:objective_definition}

The aim here is straightforward: Automate the signal timing optimization process just with video data. Specifically, we set out to:

\begin{itemize}
    \item \textbf{Data Collection \& Intersection Selection:} Identify representative signalized and non-signalized intersections in Mangalore for a comparative case study, with practical field-feasibility in mind.
    
    \item \textbf{Automated Traffic Analysis:} Build a pipeline using YOLO to automatically count vehicles from recorded video and convert them to Passenger Car Units (PCU).
    
    \item \textbf{Signal Timing Optimization:} Use the extracted PCUs to compute cycle length and green splits for each approach, primarily via Webster's method.
    
    \item \textbf{Validation \& Visualization:} Validate results with clear metrics (e.g., delay) and produce intuitive phase diagrams to explain the recommended cycle.
\end{itemize}

\section{Selection of Study Intersections}
\label{sec:intersection_selection}

We started by shortlisting intersections in Mangalore using Google Maps and field visits. We were tryinh to have variety in the intersections we selected.

Two intersections were selected:
\begin{itemize}
    \item \textbf{Jyoti Circle} -- A three-arm junction, manually controlled by traffic police. This has a problem there is a bariicade like in the photo. There were no strThis is a signalized intersection.
    \item \textbf{Hampankatta Circle} -- A four-arm intersection with signals. This has a problem there is a bariicade like in the photo. There were no straight movements allowed from the east-approach and west-approach.
\end{itemize}

These locations are both practical to study and representative of common urban conditions and had some challenges like the bariicade and the lack of straight movements.

\begin{figure}[H]
    \centering
    \begin{subfigure}{0.48\textwidth}
        \includegraphics[width=\linewidth,height=0.55\textwidth,keepaspectratio]{images/1/jyoticircle_intersection1.jpg}
        \caption{Jyoti Circle map illustrating three-arm geometry and surrounding approaches layout.}
        \label{fig:jyoti_map}
    \end{subfigure}\hfill
    \begin{subfigure}{0.48\textwidth}
        \includegraphics[width=\linewidth,height=0.55\textwidth,keepaspectratio]{images/1/jyoti_intersection_field_image_withcamerasetup.jpg}
        \caption{Jyoti Circle field setup using Insta360 camera for traffic coverage.}
        \label{fig:jyoti_field_setup}
    \end{subfigure}
    
    \vspace{0.5em}
    
    \begin{subfigure}{0.48\textwidth}
        \includegraphics[width=\linewidth,height=0.55\textwidth,keepaspectratio]{images/2/hampanakatta_intersection2.jpg}
        \caption{Hampankatta Circle map depicting four-arm geometry and approach layout details.}
        \label{fig:hampankatta_map}
    \end{subfigure}\hfill
    \begin{subfigure}{0.48\textwidth}
        \includegraphics[width=\linewidth,height=0.55\textwidth,keepaspectratio]{images/2/hampankatta_field_image.jpg}
        \caption{Hampankatta Circle field photograph showing barricades and monitoring equipment placement.}
        \label{fig:hampankatta_field}
    \end{subfigure}
    \caption{Study locations and setups for Jyoti and Hampankatta intersections overview.}
\end{figure}

\section{Field Data Collection}
\label{sec:field_data}

\subsection{Initial Plan}
We first planned to deploy multiple tripod-mounted cameras to cover each leg of both intersections.

\subsection{Challenges Faced}
Fieldwork rarely goes exactly to plan. In our case:
\begin{itemize}
    \item Continuous heavy rainfall
    \item Limited availability of tripods
\end{itemize}

\subsection{Revised Data Collection Setup}
It delayed the project by 2 weeks. Later We got a better idea to use a single 360° Insta360 camera per site. That gave us:
\begin{itemize}
    \item Complete intersection coverage
    \item Reduced manpower
    \item Lower risk of missing turning movements
\end{itemize}

Each intersection was recorded for 30 minutes and the footage processed for analysis.

\section{Signal Control Assumptions}
\label{sec:signal_assumptions}

To keep every optimization path consistent (Webster, Machine Learning, and SUMO), we locked in a common set of assumptions before running any calculations:

\subsection{Saturation Flow Baseline}
\begin{itemize}
    \item \textbf{Base capacity:} 1,800 Passenger Car Units (PCU) per hour per lane, simplified from the IRC SP:41 expression $s = 525 \times W$ for approach widths 3.5 m wide.
    \item \textbf{Default lanes per approach:} Two lanes unless field notes say otherwise.
    \item \textbf{Flow ratio cap:} The combined critical flow ratio $Y$ stays below 0.95 to avoid runaway cycle lengths.
\end{itemize}

\subsection{Phase Design Logic}
\begin{itemize}
    \item \textbf{Opposite approaches pair up:} Northbound with Southbound, Eastbound with Westbound. They never overlap.
    \item \textbf{Exclusive phases only:} When the north-south group has green, the east-west group stays red, and vice versa.
    \item \textbf{Sequence:} NS green $\rightarrow$ NS yellow $\rightarrow$ all red $\rightarrow$ EW green $\rightarrow$ EW yellow $\rightarrow$ all red, then repeat.
    \item \textbf{T-junction tweak:} Sites like Jyoti Circle (no east arm) run only the phases that match the physical approaches.
    \item \textbf{All movements signalized:} Left turns, throughs, and right turns all see the signal; no free turns are modeled in the current build to keep validation simple.
\end{itemize}

\subsection{Timing Constants}
\begin{itemize}
    \item \textbf{Amber time:} Fixed at 3 seconds per phase.
    \item \textbf{All-red interval:} Fixed at 2 seconds between phases for clearance.
    \item \textbf{Lost time:} Totals 12 seconds for a two-phase plan (startup plus clearance).
    \item \textbf{Cycle bounds:} 60-second minimum and 120-second maximum in both Webster and ML outputs.
\end{itemize}

\section{Analysis of both intersections}
\label{sec:case_study}

The system adapts to different intersection types by detecting whether a site is 3-way (T-junction) or 4-way (crossroads). It then generates the right network, routing logic, and timing approach for that geometry.

\subsection{Jyoti Circle (3-Arm T-Junction)}

Total PCU was about 6,802 with a strongly dominant NS flow. The system detected a 3-way layout (no East approach) and:
\begin{itemize}
    \item Generated appropriate T-junction network with only NB, SB, and WB approaches
    \item Applied T-junction routing logic (no through movements where not applicable)
    \item Predicted minimal cycle length ($\sim$60 s) appropriate for the traffic volume
    \item Allocated green split skewed toward the dominant direction (NS $\approx$ 36.5 s vs W $\approx$ 9.8 s)
\end{itemize}

\subsection{Hampankatta Circle (4-Arm Intersection)}

Total PCU reached 10,680 with heavy bus traffic. The system detected a 4-way intersection and:
\begin{itemize}
    \item Generated complete 4-way network with all approaches (NB, SB, EB, WB)
    \item Applied standard routing logic with through movements available
    \item Predicted longer cycle length ($\sim$92 s) to handle higher saturation
    \item Allocated balanced green splits according to proportional PCU distribution
\end{itemize}

\section{Validation and Interpretation}
\label{sec:validation}

We cross-verified signal timing calculations against IRC:106-1990 standards for signalized intersections

We then validated plans in SUMO microsimulation:
\begin{itemize}
    \item Dynamic network generation for 3-way and 4-way intersections based on detected approaches
    \item Traffic light program creation from Webster signal timing plans
    \item Route generation based on PCU values from field data
    \item Running simulations under realistic traffic conditions
    \item Extracting performance metrics including average delay, waiting time, travel time, throughput, and time loss
\end{itemize}

The sumo-simulation provided detailed metrics supporting the Webster-based signal plans:
\begin{itemize}
    \item Average delay per vehicle
    \item Average waiting time at intersections
    \item Average travel time
    \item Vehicle throughput
    \item Total time loss
\end{itemize}

In short, the microsimulation backs up the effectiveness of the optimized timings under realistic conditions.
\newpage