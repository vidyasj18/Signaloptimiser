\chapter{Abstract}
\label{ch:abstract}

\section{Summary of Objectives}
\label{sec:objectives}

This project is about automating stoplight timing at two typical junctions in Mangalore - using a mix of Webster method, ML and SUMO Simulation to compare and analyse with some time based and vehicle based metrics.

\section{Key Design Features}
\label{sec:design_features}

\begin{itemize}
    \item \textbf{Diverse sites:} Jyoti Circle (3-arm, signalized) and Hampankatta Circle (4-arm, manually controlled) capture the range of urban geometries and operating styles in Mangalore.
    \item \textbf{Single-sensor field capture:} A lone Insta360 camera per site provided full-junction coverage, reducing manpower without compromising accuracy even during adverse weather.
    \item \textbf{Automated demand estimation:} YOLOv8 counts, converted to PCUs via IRC factors, replaced manual tallies and feed directly into optimization workflows.
    \item \textbf{Dual optimization engines:} Webster's method supplies deterministic baselines, while Random Forest regressors (cycle and green splits) learn context-aware adjustments from synthetic teacher data.
    \item \textbf{Integrated tooling:} A Streamlit interface orchestrates detection, optimization, and SUMO simulation, enabling side-by-side ML vs Webster plan reviews without scripting overhead.
    \item \textbf{Traceable outputs:} Phase diagrams, performance tables, and archived SUMO artifacts document every recommendation from raw data through validation.
\end{itemize}

\section{Overview}
\label{sec:overview}

Urban intersections dictate corridor performance; in Mangalore, swelling volumes have exposed legacy signal plans that no longer reflect measured demand. Our workflow begins with video-driven PCU extraction, then branches into Webster and ML optimizers whose outputs are stress-tested in SUMO under identical conditions. This closed loop translates raw detections into vetted signal plans while highlighting where analytical versus learned strategies excel.

\section{Findings}
\label{sec:findings}

\begin{itemize}
    \item \textbf{Context drives performance:} At Jyoti Circle (3-way, 6,802~PCU/hr, 3.5:1 NS:W), Webster cut average delay by 2.9\%, travel time by 2.4\%, and depart delay by 21.1\% versus ML. At Hampankatta Circle (4-way, 10,680~PCU/hr, near-balanced), ML lowered delay by 6.4\%, travel time by 5.7\%, and time loss by 7.1\% versus Webster.
    \item \textbf{Validation matters:} SUMO simulations with identical demand confirmed both signal plans are feasible and quantified tradeoffs in throughput, waiting time, and travel time.
    \item \textbf{Method guidance:} Webster's analytical proportional splits suit lower-volume, asymmetric layouts; ML's learned splits excel in high-volume, four-arm intersections where multi-approach interactions dominate.
    \item \textbf{Implementation-ready assets:} Phase diagrams, timing tables, and archived SUMO networks/signal programs provide a full trace from detection to deployment for either method.
\end{itemize}

\newpage

