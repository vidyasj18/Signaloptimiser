\chapter{Traffic Signal Optimization with YOLO Detection}
\label{ch:yolo_optimization}

\section{Objectives}
\label{sec:yolo_objectives}

The project aimed to build a two-stage system for traffic signal optimization:
\begin{itemize}
    \item \textbf{PCU Calculator:} An application that uses YOLOv8 to process uploaded intersection videos, detect and count vehicles, and export the calculated Passenger Car Units (PCU) for each approach to a JSON file.
    \item \textbf{Signal Timer:} A separate script that ingests the PCU data and applies Webster's method to compute optimal green times and generate an exclusive phase timeline visualization.
\end{itemize}

\section{YOLO (You Only Look Once)}
\label{sec:yolo_description}

YOLO is a high-performance real-time object detection framework commonly employed in intelligent traffic management systems. It processes complete images in a single evaluation, which enables rapid and accurate identification of multiple object classes such as cars, buses, two-wheelers, and trucks. In traffic applications, YOLO supports vehicle detection, classification, counting, and movement tracking using video feeds. Its operational efficiency and reliability make it suitable for tasks related to congestion monitoring, signal timing optimization, and automated incident detection within modern traffic analysis environments.

\section{Workflow}
\label{sec:yolo_workflow}

Our workflow was successfully implemented in three distinct stages:

\subsection{Stage 1: Detection \& PCU Calculation (Streamlit App)}

Developed a Streamlit application (\texttt{main.py}) where a user can upload a recorded video for each of the 3 or 4 approaches of an intersection.
\begin{itemize}
    \item Integrated YOLOv8 for vehicle detection and tracking in the uploaded videos.
    \item Implemented inbound vehicle counting using virtual stoplines and ROI masks to ensure accurate counts.
    \item Converted the final vehicle counts to PCU values using IRC-style default factors (e.g., bus=3, car=1, motorcycle=0.5).
    \item The application's final output is a single JSON file (\texttt{outputs/intersection\_summary.json}) containing the total PCU for each approach (N, S, E, W).
\end{itemize}

\subsection{Stage 2: Signal Timing \& Visualization (Integrated in Streamlit)}

The Streamlit application integrates Webster signal timing calculation directly into the workflow:
\begin{itemize}
    \item Loads PCU data from the \texttt{outputs/intersection\_summary.json} file created in Stage 1.
    \item Applies Webster's method to compute optimal signal timings based on IRC:106-1990 standards.
    \item Calculates the effective green times, applies fixed lost times (12s), amber (3s), and all-red (2s) intervals, and produces per-phase exclusive timelines.
    \item Saves the Webster-based signal plan as JSON file (\texttt{webster\_signal\_plan.json}).
    \item Generates and displays a Plotly phase diagram visualizing the NS vs EW exclusive phases, ensuring only one phase group is green at a time.
\end{itemize}

\subsection{Stage 3: SUMO Simulation \& Validation (Integrated in Streamlit)}

The Streamlit application integrates SUMO simulation validation into the workflow:
\begin{itemize}
    \item \textbf{Network Generation:} Dynamically creates SUMO network files (.nod.xml, .edg.xml) using netconvert, automatically detecting intersection type (3-way T-junction or 4-way intersection) from signal plans.
    \item \textbf{Traffic Light Programs:} Converts signal timing plans into SUMO traffic light phase definitions (.add.xml), ensuring phase state strings match the actual connection order from the generated network.
    \item \textbf{Route Generation:} Creates vehicle routes (.rou.xml) based on PCU values, with intelligent routing logic that adapts to T-junction geometry (no through movements where not applicable).
    \item \textbf{Simulation Execution:} Runs SUMO simulation for the Webster-based signal plan under realistic traffic conditions (3600 seconds, vehicle flows based on detected PCU values).
    \item \textbf{Performance Extraction:} Extracts comprehensive metrics including average delay, waiting time, travel time, time loss, and throughput from SUMO tripinfo outputs.
    \item \textbf{Results Display:} Displays detailed performance metrics in the Streamlit interface, providing immediate feedback on signal timing effectiveness.
\end{itemize}

\section{Results and Analysis}
\label{sec:yolo_results}

We successfully processed two key intersections using the integrated Streamlit workflow combining YOLO detection, Webster optimization, and SUMO validation.

\subsection{Intersection 1: Jyoti Circle (3-Approach Y-Junction)}

\textbf{Analysis:} The total traffic (6,802 PCU) resulted in a minimal 60-second cycle using Webster's method. The demand was highly unbalanced (NS: 5,302 PCU vs. W: 1,500 PCU, a 3.5:1 ratio). Webster's method correctly allocated highly skewed green time (36.53s for NS vs. 9.86s for W, a 3.7:1 ratio), proving its ability to prioritize high-demand routes proportionally.

\subsection{Intersection 2: Hampankatta Circle (4-Approach)}

\textbf{Analysis:} This intersection had 57\% higher total traffic (10,680 PCU), with high PCU values (W=3480) indicating heavy bus routes. Webster's method correctly identified that the 60s minimum was insufficient and calculated a longer 92.07s cycle to handle this high saturation.

Furthermore, Webster's method demonstrated proportional splitting within a phase. The EW phase's demand was uneven (W: 3480 vs. E: 1560, a 2.2:1 ratio). The method correctly assigned nearly double the green time to the Westbound approach (24.76s) compared to the Eastbound (12.53s), maintaining proportional allocation based on demand.

\section{Comparative Analysis}
\label{sec:yolo_comparative}

\begin{table}[h]
\centering
\caption{Comparative Analysis of Two Intersections}
\label{tab:comparative_analysis}
\begin{tabular}{|l|l|l|}
\hline
\textbf{Metric} & \textbf{Jyoti Circle} & \textbf{Hampankatta} \\
\hline
Type & 3-Approach (T-Junction) & 4-Approach (Crossroads) \\
Total PCU & 6,802 & 10,680 (57\% higher) \\
Demand Balance & Highly Unbalanced (3.5:1) & Relatively Balanced (1.1:1) \\
Predicted Cycle & 60.44 s (Minimal) & 92.07 s (Extended) \\
Total NS Green & 36.53 s & 42.77 s \\
Total EW Green & 9.86 s & 37.29 s \\
Green Time Split & Highly Skewed (3.7:1) & Balanced (1.1:1) \\
\hline
\end{tabular}
\end{table}

\subsection{SUMO Simulation Validation Results}

To provide objective validation of the Webster-based signal timing approach, the signal plan was tested in SUMO microsimulation under realistic traffic conditions. The simulation was conducted for a 3-way T-junction (Jyoti Circle configuration) with the following setup:

\textbf{Simulation Parameters:}
\begin{itemize}
    \item Duration: 3600 seconds (1 hour)
    \item Traffic Demand: Based on detected PCU values from YOLO analysis
    \item Network: Dynamically generated 3-way intersection with 4 nodes, 6 edges, 7 connections
    \item Vehicle Type: Standard cars (5m length, max speed 50 km/h)
    \item Signal Plan: Webster-based timing with optimal cycle length and green splits
\end{itemize}

\textbf{Performance Metrics:}

The SUMO simulation provides comprehensive performance metrics that validate the effectiveness of the Webster-based signal timing:
\begin{itemize}
    \item \textbf{Average Delay:} Measures waiting time at the intersection per vehicle
    \item \textbf{Average Waiting Time:} Time vehicles spend completely stopped
    \item \textbf{Average Travel Time:} Total time from network entry to exit
    \item \textbf{Average Time Loss:} Difference from free-flow travel time
    \item \textbf{Vehicle Throughput:} Number of vehicles successfully completing trips
    \item \textbf{Total Delay and Waiting Time:} Aggregate measures of intersection performance
\end{itemize}

\textbf{Key Findings from SUMO Validation:}
\begin{itemize}
    \item The Webster-based approach produces effective signal timing plans validated in realistic traffic scenarios with vehicle interactions, queuing, and stochastic arrival patterns.
    \item The simulation accounts for complex traffic dynamics that analytical formulas cannot capture, providing objective evidence of signal plan effectiveness.
    \item The integrated workflow (YOLO detection → Webster calculation → SUMO validation) provides a complete, validated solution for traffic signal optimization.
    \item The Streamlit web application makes the entire process accessible, enabling engineers and planners to validate signal timings without requiring command-line expertise.
\end{itemize}

\newpage

