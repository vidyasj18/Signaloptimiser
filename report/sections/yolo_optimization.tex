\chapter{Traffic Signal Optimization with YOLO Detection}
\label{ch:yolo_optimization}

\section{Objectives}
\label{sec:yolo_objectives}

The primary objective of the YOLO detection component is to capture and process traffic data from video feeds:
\begin{itemize}
    \item \textbf{Vehicle Detection \& Classification:} Use YOLOv8 to detect and classify vehicles (cars, buses, motorcycles, trucks) in uploaded traffic videos for each intersection approach.
    \item \textbf{Vehicle Counting:} Count vehicles crossing virtual stoplines using tracking algorithms to ensure accurate inbound vehicle counts.
    \item \textbf{PCU Conversion:} Convert raw vehicle counts to Passenger Car Units (PCU) using IRC standards (car=1, bus=3, motorcycle=0.5, truck=2.5) and export the results to a JSON file (\texttt{outputs/intersection\_summary.json}).
\end{itemize}

The YOLO component serves as the data capture stage of the overall system. The captured PCU data is then used by subsequent components (Webster's method and SUMO simulation) for signal timing optimization and validation. YOLO itself does not perform optimization—it only provides the traffic flow data necessary for optimization algorithms.

\section{YOLO (You Only Look Once)}
\label{sec:yolo_description}

YOLO is a fast, real-time object detector. It processes an entire image in one go, which makes it well-suited to spotting multiple vehicle classes—cars, buses, two-wheelers, trucks—quickly and reliably. In our context, that means accurate detection, classification, and counting using video feeds—exactly what you want for signal timing and congestion analysis.

\section{YOLO Detection Process}
\label{sec:yolo_workflow}

The YOLO detection process is implemented through a Streamlit application (\texttt{main.py}) that processes traffic videos and extracts vehicle data:

\subsection{Video Processing \& Vehicle Detection}

The system processes uploaded videos for each intersection approach (3 or 4 approaches depending on intersection type):
\begin{itemize}
    \item \textbf{Video Upload:} Users upload traffic videos for each approach (Northbound, Southbound, Eastbound, Westbound) through the Streamlit interface.
    \item \textbf{YOLOv8 Integration:} The YOLOv8 model processes each video frame to detect and classify vehicles into categories: cars, buses, motorcycles, trucks, and bicycles.
    \item \textbf{Vehicle Tracking:} A tracking algorithm assigns unique IDs to detected vehicles and tracks them across frames to avoid double-counting.
    \item \textbf{ROI Masks \& Stoplines:} Region-of-Interest (ROI) masks and virtual stoplines are used to ensure only inbound vehicles crossing the stopline are counted, improving accuracy.
\end{itemize}

\subsection{Vehicle Counting \& PCU Conversion}

After detection, the system performs counting and conversion:
\begin{itemize}
    \item \textbf{Inbound Vehicle Counting:} Vehicles crossing the virtual stopline in the inbound direction are counted, with minimum frame requirements to filter out false detections.
    \item \textbf{Vehicle Classification:} Each detected vehicle is classified by type (car, bus, motorcycle, truck, bicycle) based on YOLO's classification output.
    \item \textbf{PCU Conversion:} Raw vehicle counts are converted to Passenger Car Units (PCU) using IRC:106-1990 standards:
    \begin{itemize}
        \item Car: 1.0 PCU
        \item Bus: 3.0 PCU
        \item Motorcycle: 0.5 PCU
        \item Truck: 2.5 PCU
        \item Bicycle: 0.2 PCU
    \end{itemize}
    \item \textbf{Data Export:} The final output is a JSON file (\texttt{outputs/intersection\_summary.json}) containing PCU totals for each approach (N, S, E, W), along with detailed vehicle counts by type.
\end{itemize}

Figure~\ref{fig:yolo_output} shows the Streamlit interface displaying YOLO detection results, including vehicle counts by type, PCU calculations, and per-approach summaries. The interface provides real-time feedback on detection accuracy and allows users to review and validate the detected vehicle counts.

\begin{figure}[H]
    \centering
    \includegraphics[width=0.95\textwidth]{images/intersectoin1yolooutput.png}
    \caption{Streamlit interface showing YOLO detection results for Intersection 1 (Jyoti Circle). The output displays vehicle counts by type (cars, buses, motorcycles, trucks) for each approach, PCU calculations using IRC standards, and a summary table showing total PCU values per approach. The interface allows users to review detection accuracy before exporting the data for signal optimization.}
    \label{fig:yolo_output}
\end{figure}

\section{YOLO Detection Results}
\label{sec:yolo_results}

We processed traffic videos for two key intersections using YOLO detection. The system successfully detected, classified, and counted vehicles across all approaches, providing detailed traffic composition data.

\subsection{Vehicle Type Distribution}

Figure~\ref{fig:vehicle_distribution} shows the vehicle type distribution across all approaches, demonstrating the diversity of traffic composition detected by YOLO. The chart reveals the relative proportions of different vehicle types (cars, buses, motorcycles, trucks) at each intersection approach, which directly influences PCU calculations.

\begin{figure}[H]
    \centering
    \includegraphics[width=0.9\textwidth]{images/vehicle_type_distribution.png}
    \caption{Vehicle type distribution across approaches (NB, SB, EB, WB) from YOLO detection results. The chart shows the count of cars, buses, motorcycles, and trucks detected in each approach, which are then converted to PCU values using IRC standards (car=1, bus=3, motorcycle=0.5, truck=2.5). The distribution reveals that cars dominate traffic flow, with significant motorcycle presence and varying bus/truck counts across approaches.}
    \label{fig:vehicle_distribution}
\end{figure}

\subsection{Intersection 1: Jyoti Circle (3-Approach Y-Junction)}

\textbf{YOLO Detection Results:}
\begin{itemize}
    \item \textbf{Total PCU Detected:} 6,802 PCU/hr
    \item \textbf{Approach Breakdown:}
    \begin{itemize}
        \item Northbound (NB): 2,542 PCU/hr
        \item Southbound (SB): 2,760 PCU/hr
        \item Westbound (WB): 1,500 PCU/hr
        \item Eastbound (EB): 0 PCU/hr (no approach)
    \end{itemize}
    \item \textbf{Traffic Characteristics:} Highly asymmetric flow distribution (NS: 5,302 PCU vs W: 1,500 PCU, ratio 3.5:1)
    \item \textbf{Vehicle Composition:} Dominated by cars and motorcycles, with moderate bus and truck presence
\end{itemize}

\textbf{Data Quality:} YOLO successfully detected and classified vehicles across all three active approaches, providing accurate counts for PCU conversion. The detection captured the asymmetric nature of traffic flow, with significantly higher volumes in the north-south direction compared to westbound.

\subsection{Intersection 2: Hampankatta Circle (4-Approach)}

\textbf{YOLO Detection Results:}
\begin{itemize}
    \item \textbf{Total PCU Detected:} 10,680 PCU/hr (57\% higher than Jyoti Circle)
    \item \textbf{Approach Breakdown:}
    \begin{itemize}
        \item Northbound (NB): 2,880 PCU/hr
        \item Southbound (SB): 2,760 PCU/hr
        \item Eastbound (EB): 1,560 PCU/hr
        \item Westbound (WB): 3,480 PCU/hr (highest volume)
    \end{itemize}
    \item \textbf{Traffic Characteristics:} Relatively balanced flow distribution (NS: 5,640 PCU vs EW: 5,040 PCU, ratio 1.1:1)
    \item \textbf{Vehicle Composition:} Heavy bus movements detected, particularly in the westbound direction (3,480 PCU), indicating a major transit route
\end{itemize}

\textbf{Data Quality:} YOLO detection successfully captured all four approaches, with accurate classification of vehicle types. The detection revealed significant bus traffic in the westbound direction, which contributes substantially to the total PCU due to the 3.0 PCU factor for buses.

\section{Data Output \& Export}

The YOLO detection process produces structured data outputs:

\subsection{JSON Output Format}

The primary output is \texttt{outputs/intersection\_summary.json}, containing:
\begin{itemize}
    \item Per-approach vehicle counts by type (car, bus, motorcycle, truck, bicycle)
    \item Total PCU values for each approach
    \item Video processing metadata (duration, frame count, FPS)
    \item Detection confidence metrics
\end{itemize}

\subsection{CSV Output Files}

Additionally, per-approach CSV files are generated (\texttt{NB\_summary.csv}, \texttt{SB\_summary.csv}, etc.), containing:
\begin{itemize}
    \item Frame-by-frame vehicle detection data
    \item Time-series vehicle counts
    \item Detection timestamps
    \item Vehicle type breakdowns per frame
\end{itemize}

These outputs provide the foundation for subsequent signal optimization algorithms, which use the PCU data to calculate optimal signal timing plans.

\newpage

