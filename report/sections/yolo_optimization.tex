\chapter{Traffic Signal Optimization with YOLO Detection}
\label{ch:yolo_optimization}

\section{Objectives}
\label{sec:yolo_objectives}

We built a simple two-stage system to optimize signals:
\begin{itemize}
    \item \textbf{PCU Calculator:} A tool that uses YOLOv8 to process uploaded videos, detect and count vehicles, and export Passenger Car Units (PCU) for each approach to a JSON file.
    \item \textbf{Signal Timer:} A script that reads the PCU data, applies Webster’s method, and outputs optimal green times plus a clear phase timeline.
\end{itemize}

\section{YOLO (You Only Look Once)}
\label{sec:yolo_description}

YOLO is a fast, real-time object detector. It processes an entire image in one go, which makes it well-suited to spotting multiple vehicle classes—cars, buses, two-wheelers, trucks—quickly and reliably. In our context, that means accurate detection, classification, and counting using video feeds—exactly what you want for signal timing and congestion analysis.

\section{Workflow}
\label{sec:yolo_workflow}

The workflow runs in three straightforward stages:

\subsection{Stage 1: Detection \& PCU Calculation (Streamlit App)}

A Streamlit app (\texttt{main.py}) lets users upload videos for each of the 3 or 4 approaches.
\begin{itemize}
    \item Integrated YOLOv8 for vehicle detection and tracking in the uploaded videos.
    \item Implemented inbound vehicle counting using virtual stoplines and ROI masks to ensure accurate counts.
    \item Converted counts to PCU using IRC-style factors (e.g., bus=3, car=1, motorcycle=0.5).
    \item Final output: a single JSON file (\texttt{outputs/intersection\_summary.json}) with PCU totals for each approach (N, S, E, W).
\end{itemize}

\subsection{Stage 2: Signal Timing \& Visualization (Integrated in Streamlit)}

The app then computes timings using Webster’s method:
\begin{itemize}
    \item Loads PCU data from the \texttt{outputs/intersection\_summary.json} file created in Stage 1.
    \item Applies Webster’s method (aligned with IRC:106-1990) to compute optimal timings.
    \item Calculates effective greens and applies fixed lost times (12s), amber (3s), and all-red (2s), producing per-phase timelines.
    \item Saves the Webster-based signal plan as JSON file (\texttt{webster\_signal\_plan.json}).
    \item Generates and displays a Plotly phase diagram visualizing the NS vs EW exclusive phases, ensuring only one phase group is green at a time.
\end{itemize}

\subsection{Stage 3: SUMO Simulation \& Validation (Integrated in Streamlit)}

Finally, the app validates in SUMO:
\begin{itemize}
    \item \textbf{Network Generation:} Dynamically creates SUMO network files (.nod.xml, .edg.xml) using netconvert, automatically detecting intersection type (3-way T-junction or 4-way intersection) from signal plans.
    \item \textbf{Traffic Light Programs:} Converts signal timing plans into SUMO traffic light phase definitions (.add.xml), ensuring phase state strings match the actual connection order from the generated network.
    \item \textbf{Route Generation:} Creates vehicle routes (.rou.xml) based on PCU values, with intelligent routing logic that adapts to T-junction geometry (no through movements where not applicable).
    \item \textbf{Simulation Execution:} Runs SUMO simulation for the Webster-based signal plan under realistic traffic conditions (3600 seconds, vehicle flows based on detected PCU values).
    \item \textbf{Performance Extraction:} Extracts comprehensive metrics including average delay, waiting time, travel time, time loss, and throughput from SUMO tripinfo outputs.
    \item \textbf{Results Display:} Displays detailed performance metrics in the Streamlit interface, providing immediate feedback on signal timing effectiveness.
\end{itemize}

\section{Results and Analysis}
\label{sec:yolo_results}

Using the integrated workflow (YOLO + Webster + SUMO), we processed two key intersections.

\subsection{Intersection 1: Jyoti Circle (3-Approach Y-Junction)}

\textbf{Analysis:} Total traffic (6,802 PCU) produced a minimal 60-second cycle. Demand was highly unbalanced (NS: 5,302 vs W: 1,500 PCU, ~3.5:1). Webster’s method assigned green time accordingly (36.53s NS vs 9.86s W, ~3.7:1), prioritizing the high-demand leg.

\subsection{Intersection 2: Hampankatta Circle (4-Approach)}

\textbf{Analysis:} Total traffic was 57\% higher (10,680 PCU), with heavy bus movements (W=3480). The 60s minimum wouldn’t cut it; the computed cycle was 92.07s to handle the higher saturation.

Within the EW phase, demand was uneven (W: 3480 vs E: 1560, ~2.2:1). The split reflected that: ~24.76s W vs ~12.53s E—nearly 2:1—keeping allocation proportional.

\section{Comparative Analysis}
\label{sec:yolo_comparative}

\begin{table}[h]
\centering
\caption{Comparative Analysis of Two Intersections}
\label{tab:comparative_analysis}
\begin{tabular}{|l|l|l|}
\hline
\textbf{Metric} & \textbf{Jyoti Circle} & \textbf{Hampankatta} \\
\hline
Type & 3-Approach (T-Junction) & 4-Approach (Crossroads) \\
Total PCU & 6,802 & 10,680 (57\% higher) \\
Demand Balance & Highly Unbalanced (3.5:1) & Relatively Balanced (1.1:1) \\
Predicted Cycle & 60.44 s (Minimal) & 92.07 s (Extended) \\
Total NS Green & 36.53 s & 42.77 s \\
Total EW Green & 9.86 s & 37.29 s \\
Green Time Split & Highly Skewed (3.7:1) & Balanced (1.1:1) \\
\hline
\end{tabular}
\end{table}

\subsection{SUMO Simulation Validation Results}

To objectively validate the Webster-based plan, we ran SUMO microsimulations for a 3-way T-junction (Jyoti Circle configuration) using:

\textbf{Simulation Parameters:}
\begin{itemize}
    \item Duration: 3600 seconds (1 hour)
    \item Traffic Demand: Based on detected PCU values from YOLO analysis
    \item Network: Dynamically generated 3-way intersection with 4 nodes, 6 edges, 7 connections
    \item Vehicle Type: Standard cars (5m length, max speed 50 km/h)
    \item Signal Plan: Webster-based timing with optimal cycle length and green splits
\end{itemize}

\textbf{Performance Metrics:}

SUMO outputs comprehensive metrics to validate timing effectiveness:
\begin{itemize}
    \item \textbf{Average Delay:} Measures waiting time at the intersection per vehicle
    \item \textbf{Average Waiting Time:} Time vehicles spend completely stopped
    \item \textbf{Average Travel Time:} Total time from network entry to exit
    \item \textbf{Average Time Loss:} Difference from free-flow travel time
    \item \textbf{Vehicle Throughput:} Number of vehicles successfully completing trips
    \item \textbf{Total Delay and Waiting Time:} Aggregate measures of intersection performance
\end{itemize}

\textbf{Key Findings from SUMO Validation:}
\begin{itemize}
    \item The Webster-based approach yields effective signal timing plans in realistic scenarios with queues, interactions, and random arrivals.
    \item Simulation captures dynamics that pure formulas miss, adding objective evidence.
    \item The end-to-end flow (YOLO → Webster → SUMO) is complete and validated.
    \item The Streamlit app keeps it accessible—no command line required.
\end{itemize}

\newpage

