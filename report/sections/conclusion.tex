\chapter{Conclusion}
\label{ch:conclusion}

This Mini Project successfully developed a data-driven framework to optimize traffic signal timing at urban intersections, integrating classical traffic engineering with modern computational techniques. Using YOLOv8 for automated vehicle detection and PCU estimation combined with Webster's Method and Machine Learning models such as Linear Regression and Random Forest, the system effectively predicts optimal cycle lengths and green time splits.

The project's key innovation lies in creating realistic training datasets that incorporate real-world traffic patterns—time-of-day effects, weather impacts, special events, and day-of-week variations—enabling ML models to learn beyond Webster's theoretical calculations. This approach ensures the models can adapt to actual traffic conditions that traditional formulas cannot capture.

Validation through case studies at Jyoti Circle (3-way T-junction) and Hampankatta Circle (4-way intersection) demonstrated the model's adaptability by adjusting signal phases to prioritize heavy or unbalanced traffic flows. The system dynamically generates appropriate networks and routing logic for different intersection types, proving its versatility.

Most importantly, SUMO microsimulation validation provided objective evidence of the ML-enhanced approach's superiority. Under identical traffic conditions, the ML-based signal plan outperformed Webster's method in 4 out of 5 performance metrics, achieving a 1.3\% reduction in average vehicle delay (8.84s vs 8.96s). This validation confirms that the ML model's training on real-world patterns translates to measurable performance improvements in realistic traffic scenarios.

The project also produced comprehensive outputs including signal phasing diagrams, SUMO network files, traffic light programs, and detailed performance comparisons. These outputs provide city engineers and planners with actionable, validated recommendations for signal timing optimization.

By comparing traditional Webster formula-based timing with data-driven ML predictions validated through SUMO simulation, this work establishes a robust foundation for intelligent traffic signal control systems. The framework is designed to be compatible with real-world sensor data, enabling future retraining and deployment with actual intersection measurements. This project marks an important step towards intelligent transportation systems that combine theoretical rigor with practical adaptability, ultimately supporting safer, more efficient, and sustainable urban mobility.

\section{Key Achievements}
\label{sec:achievements}

\begin{enumerate}
    \item Successfully developed an end-to-end automated framework for traffic signal optimization
    \item Created realistic synthetic datasets incorporating six real-world factors (time-of-day, weather, events, day-of-week, directional bias, capacity variations)
    \item Implemented ML models that outperform traditional Webster method in SUMO simulation
    \item Validated approach for both 3-way and 4-way intersections with dynamic adaptation
    \item Demonstrated 1.3\% improvement in average vehicle delay through microsimulation
    \item Created reusable, scalable framework compatible with real sensor data
\end{enumerate}

\section{Future Work}
\label{sec:future_work}

\begin{enumerate}
    \item Collection of real-world traffic data for model retraining and validation
    \item Extension to more complex intersection geometries (roundabouts, multi-phase signals)
    \item Integration with real-time traffic monitoring systems
    \item Development of adaptive signal control that responds to live traffic conditions
    \item Expansion to network-level optimization coordinating multiple intersections
    \item Integration with connected vehicle technologies (V2X communication)
    \item Long-term field deployment and performance monitoring
\end{enumerate}

\section{Practical Implementation Recommendations}
\label{sec:recommendations}

\begin{enumerate}
    \item \textbf{For Jyoti Circle:} Implement ML-optimized signal plan with 60-second cycle and highly skewed green splits favoring dominant NS flow
    \item \textbf{For Hampankatta Circle:} Deploy extended 92-second cycle with balanced green time distribution
    \item \textbf{Data Collection:} Install automated vehicle counting systems for continuous model improvement
    \item \textbf{Phased Rollout:} Begin with trial period, monitor performance, adjust as needed
    \item \textbf{Performance Monitoring:} Track delay, throughput, and user satisfaction metrics
    \item \textbf{Regular Updates:} Retrain models quarterly with accumulated real-world data
\end{enumerate}

\vspace{1cm}

This project successfully demonstrates that machine learning-enhanced traffic signal optimization, when properly validated through microsimulation, offers measurable improvements over traditional methods while maintaining compatibility with established traffic engineering principles.

\newpage

