\chapter{Conclusion}
\label{ch:conclusion}

This Mini Project successfully developed a data-driven framework to optimize traffic signal timing at urban intersections, integrating classical traffic engineering with modern computational techniques. Using YOLOv8 for automated vehicle detection and PCU estimation combined with Webster's Method and SUMO microsimulation validation, the system effectively calculates optimal cycle lengths and green time splits.

The project's key innovation lies in creating an integrated Streamlit web application that combines YOLO-based vehicle detection, Webster signal timing optimization, and SUMO simulation validation in a single, user-friendly workflow. This approach makes traffic signal optimization accessible to engineers and planners without requiring extensive programming knowledge or command-line expertise.

Validation through case studies at Jyoti Circle (3-way T-junction) and Hampankatta Circle (4-way intersection) demonstrated the system's adaptability by adjusting signal phases to prioritize heavy or unbalanced traffic flows. The system dynamically generates appropriate networks and routing logic for different intersection types, proving its versatility.

Most importantly, SUMO microsimulation validation provided objective evidence of the Webster-based approach's effectiveness. The simulation accounts for vehicle interactions, queuing dynamics, and stochastic arrival patterns that analytical formulas cannot capture, providing detailed performance metrics including average delay, waiting time, travel time, and throughput.

The project also produced comprehensive outputs including signal phasing diagrams, SUMO network files, traffic light programs, and detailed performance metrics. These outputs provide city engineers and planners with actionable, validated recommendations for signal timing optimization.

By integrating YOLO-based vehicle detection, Webster formula-based timing optimization, and SUMO simulation validation through a Streamlit web application, this work establishes a robust foundation for intelligent traffic signal control systems. The framework is designed to work with real-world traffic data, enabling practical deployment with actual intersection measurements. This project marks an important step towards intelligent transportation systems that combine theoretical rigor with practical accessibility, ultimately supporting safer, more efficient, and sustainable urban mobility.

\section{Key Achievements}
\label{sec:achievements}

\begin{enumerate}
    \item Successfully developed an end-to-end automated framework for traffic signal optimization
    \item Integrated YOLO-based vehicle detection, Webster signal timing calculation, and SUMO simulation in a unified Streamlit web application
    \item Implemented Webster's method for optimal signal timing based on IRC:106-1990 standards
    \item Validated approach for both 3-way and 4-way intersections with dynamic network generation
    \item Demonstrated effective signal timing optimization through SUMO microsimulation validation
    \item Created reusable, scalable framework accessible through user-friendly web interface
\end{enumerate}

\section{Future Work}
\label{sec:future_work}

\begin{enumerate}
    \item Collection of real-world traffic data for model retraining and validation
    \item Extension to more complex intersection geometries (roundabouts, multi-phase signals)
    \item Integration with real-time traffic monitoring systems
    \item Development of adaptive signal control that responds to live traffic conditions
    \item Expansion to network-level optimization coordinating multiple intersections
    \item Integration with connected vehicle technologies (V2X communication)
    \item Long-term field deployment and performance monitoring
\end{enumerate}

\section{Practical Implementation Recommendations}
\label{sec:recommendations}

\begin{enumerate}
    \item \textbf{For Jyoti Circle:} Implement Webster-optimized signal plan with 60-second cycle and highly skewed green splits favoring dominant NS flow
    \item \textbf{For Hampankatta Circle:} Deploy extended 92-second cycle with balanced green time distribution
    \item \textbf{Data Collection:} Install automated vehicle counting systems or use video-based YOLO detection for continuous traffic monitoring
    \item \textbf{Phased Rollout:} Begin with trial period, monitor performance using SUMO simulation, adjust as needed
    \item \textbf{Performance Monitoring:} Track delay, throughput, and user satisfaction metrics using SUMO validation
    \item \textbf{Regular Updates:} Recalculate signal timings periodically with updated traffic data using the Streamlit application
\end{enumerate}

\vspace{1cm}

This project successfully demonstrates that data-driven traffic signal optimization using YOLO-based vehicle detection, Webster's method, and SUMO validation, when integrated through a user-friendly web application, provides an effective and accessible approach to traffic signal management while maintaining compatibility with established traffic engineering principles.

\newpage

