\chapter{Conclusion}
\label{ch:conclusion}

This mini project built a practical, data-driven framework to optimize traffic signal timings at urban intersections—bringing classical traffic engineering together with modern tools. With YOLOv8 for automated detection and PCU estimation, Webster’s method for timing, and SUMO for validation, the system computes cycle lengths and green splits that work in the real world.

A key contribution is the integrated Streamlit app that combines YOLO-based detection, Webster timing, and SUMO validation in one workflow. That makes optimization accessible to engineers and planners without needing heavy programming or command-line tools.

Case studies at Jyoti Circle (3-way T) and Hampankatta Circle (4-way) showed the system adapting to unbalanced and heavy flows, prioritizing where it matters. It dynamically generates networks and routing logic for each geometry, which makes it versatile.

Most importantly, SUMO microsimulation provided objective evidence for the approach. By accounting for interactions, queues, and randomness that formulas miss, it yields detailed metrics like average delay, waiting time, travel time, and throughput.

We also produced practical outputs—phasing diagrams, SUMO network files, traffic light programs, and metrics—so city teams have actionable, validated recommendations.

Bringing YOLO detection, Webster timing, and SUMO validation together in a user-friendly app lays a strong foundation for intelligent signal control. The framework is built for real data and real deployment. It’s a small but meaningful step toward systems that blend rigor with day-to-day practicality—supporting safer, smoother, and more sustainable urban mobility.

\section{Key Achievements}
\label{sec:achievements}

\begin{enumerate}
    \item Successfully developed an end-to-end automated framework for traffic signal optimization
    \item Integrated YOLO-based vehicle detection, Webster signal timing calculation, and SUMO simulation in a unified Streamlit web application
    \item Implemented Webster's method for optimal signal timing based on IRC:106-1990 standards
    \item Validated approach for both 3-way and 4-way intersections with dynamic network generation
    \item Demonstrated effective signal timing optimization through SUMO microsimulation validation
    \item Created reusable, scalable framework accessible through user-friendly web interface
\end{enumerate}

\section{Future Work}
\label{sec:future_work}

\begin{enumerate}
    \item Collection of real-world traffic data for model retraining and validation
    \item Extension to more complex intersection geometries (roundabouts, multi-phase signals)
    \item Integration with real-time traffic monitoring systems
    \item Development of adaptive signal control that responds to live traffic conditions
    \item Expansion to network-level optimization coordinating multiple intersections
    \item Integration with connected vehicle technologies (V2X communication)
    \item Long-term field deployment and performance monitoring
\end{enumerate}

\section{Practical Implementation Recommendations}
\label{sec:recommendations}

\begin{enumerate}
    \item \textbf{For Jyoti Circle:} Implement Webster-optimized signal plan with 60-second cycle and highly skewed green splits favoring dominant NS flow
    \item \textbf{For Hampankatta Circle:} Deploy extended 92-second cycle with balanced green time distribution
    \item \textbf{Data Collection:} Install automated vehicle counting systems or use video-based YOLO detection for continuous traffic monitoring
    \item \textbf{Phased Rollout:} Begin with trial period, monitor performance using SUMO simulation, adjust as needed
    \item \textbf{Performance Monitoring:} Track delay, throughput, and user satisfaction metrics using SUMO validation
    \item \textbf{Regular Updates:} Recalculate signal timings periodically with updated traffic data using the Streamlit application
\end{enumerate}

\vspace{1cm}

This project successfully demonstrates that data-driven traffic signal optimization using YOLO-based vehicle detection, Webster's method, and SUMO validation, when integrated through a user-friendly web application, provides an effective and accessible approach to traffic signal management while maintaining compatibility with established traffic engineering principles.

\newpage

