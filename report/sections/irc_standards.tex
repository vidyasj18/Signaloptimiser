\chapter{IRC Standards and Assumptions}
\label{ch:irc_standards}

The system is anchored to Indian Roads Congress guidance to ensure every number in the workflow can be defended. Only the clauses we actually relied on are summarized below; unused portions of the IRC manuals are intentionally left out so the chapter stays focused.

\section{Core References}
\begin{itemize}
    \item \textbf{IRC:106-1990}: Passenger Car Unit (PCU) definitions for mixed traffic.
    \item \textbf{IRC SP:41}: Webster timing equations, saturation flow, and lost time guidance.
\end{itemize}

\section{PCU Conversion Factors}

YOLO detections are converted to PCUs using the IRC:106 ranges but collapsed to the values we actually deploy in the pipeline:
\begin{table}[H]
\centering
\caption{PCU factors applied in the project (derived from IRC:106 Table 1)}
\label{tab:irc_pcu_factors}
\begin{tabular}{|l|c|}
\hline
\textbf{Vehicle type} & \textbf{PCU used} \\
\hline
Two-wheeler & 0.5 \\
Passenger car & 1.0 \\
Bus & 3.0 \\
Truck & 3.0 \\
\hline
\end{tabular}
\end{table}
These values feed both Webster calculations and the SUMO route generation logic.

\section{Signal Timing Formulas Used}

Only three IRC SP:41 equations are part of the automation stack:
\begin{align}
C_o &= \frac{1.5L + 5}{1 - Y} &&\text{(optimum cycle length)} \label{eq:webster_cycle}\\
G_e &= C_o - L &&\text{(effective green time)} \label{eq:effective_green}\\
g_a &= \frac{y_a}{Y}(C_o - L) &&\text{(per-approach green allocation)} \label{eq:green_allocation}
\end{align}
where $Y$ is the sum of critical flow ratios and $y_a$ is the flow ratio of approach $a$.

\section{Adopted Timing Assumptions}

Directly from IRC SP:41, Section 7.6:
\begin{itemize}
    \item Amber = 3 s, all-red = 2 s $\Rightarrow$ lost time $L = 12$ s for two-phase operation.
    \item Base saturation flow = 1,800 PCU/h/lane (simplified from $s = 525W$ for widths 5.5-18 m).
    \item Practical cycle bounds used in the code: 60 s $\leq C_o \leq$ 120 s.
\end{itemize}

\section{Where Each Standard Shows Up}

\begin{table}[H]
\centering
\caption{IRC compliance touchpoints in the implementation}
\label{tab:irc_compliance}
\begin{tabular}{|l|l|}
\hline
\textbf{Guidance} & \textbf{Usage} \\
\hline
IRC:106 PCU factors & YOLO count $\rightarrow$ PCU conversion, SUMO demand generation \\
\hline
IRC SP:41 Webster formulas & Cycle length and per-phase green calculations (Section~\ref{eq:webster_cycle}--\ref{eq:green_allocation}) \\
\hline
IRC SP:41 lost time & Fixed amber/all-red durations in both Webster and ML signal plans \\
\hline
IRC SP:41 saturation flow & Sets the cap for flow ratios $y_a$ and $Y$ \\
\hline
\end{tabular}
\end{table}

\section{Documented Deviations}

\begin{itemize}
    \item \textbf{Simplified PCU values:} Buses and trucks both set to 3.0 PCU to keep the YOLO\,$\rightarrow$\,PCU pipeline deterministic when composition percentages fluctuate rapidly.
    \item \textbf{Cycle range:} Lower bound held at 60 s (common in Indian deployments) although IRC cites 120 s as the comfortable upper limit.
    \item \textbf{ML enhancements:} Contextual features (hour, weather, events) extend the deterministic IRC guidance; they augment rather than replace Webster’s baseline.
\end{itemize}

Every figure, equation, and assumption outside these bullet points was excluded from the project, so they are no longer included in this chapter.

\newpage